\usepackage{mathtools}

%Latex
	%現在の見出し番号を返す
	\newcommand{\getTitleNum}{%「%」で改行文字をコメントアウト
		\ifthenelse{\arabic{chapter}=0}{\thepart}{%
			\ifthenelse{\arabic{section}=0}{\thechapter}{%
				\ifthenelse{\arabic{subsection}=0}{\thesection}{%
					\ifthenelse{\arabic{subsubsection}=0}{\thesubsection}{%
						\ifthenelse{\arabic{subsubsubsection}=0}{\thesubsubsection}{\thesubsubsubsection}%
					}%
				}%
			}%
		}%
	}
	%textcolorの色
	\definecolor{darkpastelgreen}{rgb}{0.01, 0.75, 0.24}

%自然科学
	%汎用
	\newcommand{\ctext}[1]{\raise0.2ex\hbox{\textcircled{\scriptsize{#1}}}}

	%数学
		%汎用
		\newcommand{\as}{{\quad\textrm{as}\quad}}
		\newcommand{\st}{{\textrm{ s.t. }}}
		\DeclarePairedDelimiterX{\set}[2]{\lbrace}{\rbrace}{#1\,\delimsize\vert\,#2}
		\newcommand{\naturalNumbers}{\mathbb{N}}
		\newcommand{\integers}{\mathbb{Z}}
		\newcommand{\rationalNumbers}{\mathbb{Q}}
		\newcommand{\realNumbers}{\mathbb{R}}
		\newcommand{\complexNumbers}{\mathbb{C}}
		\newcommand{\field}{\mathbb{F}}
		\newcommand{\func}[2]{{#1}\left({#2}\right)}
		\newcommand{\argmax}{\mathop{\textrm{arg~max}}}
		\newcommand{\argmin}{\mathop{\textrm{arg~min}}}

		%集合論
		\newcommand{\range}[2]{\{#1,\dotsc,#2\}}
		\renewcommand{\complement}{\mathrm{c}}
		\newcommand{\ind}[2]{\mathbbm{1}_{#1}\left(#2\right)}
		\newcommand{\indII}[1]{\mathbbm{1}\left\{#1\right\}}

		%数論
		\newcommand{\abs}[1]{\left|#1\right|}
		\newcommand{\combi}[2]{{_{#1}\mathrm{C}_{#2}}}
		\newcommand{\perm}[2]{{_{#1}\mathrm{P}_{#2}}}
		\newcommand{\GaloisField}[1]{\mathrm{GF}\left(#1\right)}

		%解析学
		\newcommand{\sgn}[1]{\operatorname{sgn}\left(#1\right)}
		\newcommand{\cl}[1]{\operatorname{cl}#1}
		\newcommand{\Img}[1]{\operatorname{Img}\left(#1\right)}
		\newcommand{\dom}[1]{\operatorname{dom}\left(#1\right)}
		\newcommand{\norm}[1]{\left\|#1\right\|}
		\newcommand{\floor}[1]{\left\lfloor#1\right\rfloor}
		\newcommand{\ceil}[1]{\left\lceil#1\right\rceil}
		\newcommand{\expo}[1]{\exp\left(#1\right)}
		\newcommand{\sinc}{\mathop{\textrm{sinc}}}
		\newcommand{\GammaFunc}[1]{\Gamma\left(#1\right)}
			%逆三角関数
			\newcommand{\asin}[1]{\operatorname{Sin}^{-1}{#1}}
			\newcommand{\acos}[1]{\operatorname{Cos}^{-1}{#1}}
			\newcommand{\atan}[1]{\operatorname{{Tan}^{-1}}{#1}}
			\newcommand{\atanEx}[2]{\atan{\left(#1,#2\right)}}
			%微分
			\newcommand{\deriv}[3]{\frac{\operatorname{d}^{#3}#1}{\operatorname{d}{#2}^{#3}}}
			\newcommand{\derivLong}[3]{\frac{\operatorname{d}^{#3}}{\operatorname{d}{#2}^{#3}}#1}
			\newcommand{\partDeriv}[3]{\frac{\operatorname{\partial}^{#3}#1}{\operatorname{\partial}{#2}^{#3}}}
			\newcommand{\partDerivLong}[3]{\frac{\operatorname{\partial}^{#3}}{\operatorname{\partial}{#2}^{#3}}#1}
			\newcommand{\partDerivIIHetero}[3]{\frac{\operatorname{\partial}^2#1}{\partial#2\operatorname{\partial}#3}}
			\newcommand{\partDerivIIHeteroLong}[3]{{\frac{\operatorname{\partial}^2}{\partial#2\operatorname{\partial}#3}#1}}
			%積分
			\newcommand{\integ}[4]{\int_{#1}^{#2}{#3}\mathrm{d}#4}
			\newcommand{\LebInteg}[4]{\int_{#1} {#2} {#3}\left(\mathrm{d}#4\right)}

		%複素解析
		\newcommand{\conj}[1]{\overline{#1}}
		\renewcommand{\Re}[1]{{\operatorname{Re}{\left(#1\right)}}}
		\renewcommand{\Im}[1]{{\operatorname{Im}{\left(#1\right)}}}
		\newcommand{\Arg}[1]{\operatorname{Arg}{\left({#1}\right)}}
		\newcommand{\Log}[1]{\operatorname{Log}{#1}}
			%ラプラス変換
			\newcommand{\LPLC}[1]{\operatorname{\mathcal{L}}\left[#1\right]}
			\newcommand{\ILPLC}[1]{\operatorname{\mathcal{L}}^{-1}\left[#1\right]}

		%線形代数
		\newcommand{\bm}[1]{{\boldsymbol{#1}}}
		\newcommand{\Span}[1]{\operatorname{span}\left[#1\right]}
		\newcommand{\Ker}[1]{\operatorname{Ker}\left(#1\right)}
		\newcommand{\rank}[1]{\operatorname{rank}\left(#1\right)}
		\newcommand{\inprod}[2]{\left\langle#1,#2\right\rangle}
		\newcommand{\matEntry}[3]{#1\left[#2\right]\left[#3\right]}
		\newcommand{\matPart}[5]{\matEntry{#1}{#2:#3}{#4:#5}}
		\newcommand{\diag}[1]{\operatorname{diag}\left(#1\right)}
		\newcommand{\tr}[1]{\operatorname{tr}{\left(#1\right)}}
			%ベクトル
				%単位ベクトル
				\newcommand{\vix}{\bm{i}_x}
				\newcommand{\viy}{\bm{i}_y}
				\newcommand{\viz}{\bm{i}_z}

		%確率論
		\newcommand{\PDF}[2]{\operatorname{PDF}\left[#1,\;#2\right]}
		\newcommand{\Ber}[1]{\operatorname{Ber}\left(#1\right)}
		\newcommand{\Beta}[2]{\operatorname{Beta}\left(#1,#2\right)}
		\newcommand{\ExpDist}[1]{\operatorname{ExpDist}\left(#1\right)}
		\newcommand{\ErlangDist}[2]{\operatorname{ErlangDist}\left(#1,#2\right)}
		\newcommand{\PoissonDist}[1]{\operatorname{PoissonDist}\left(#1\right)}
		\newcommand{\GammaDist}[2]{\operatorname{Gamma}\left(#1,#2\right)}
		\newcommand{\cind}[2]{\ind{#1\left| #2\right.}}	%条件付き指示関数
		\renewcommand{\Pr}[1]{\operatorname{Pr}\left(#1\right)}
		\DeclarePairedDelimiterX{\cPrParen}[2]{(}{)}{#1\,\delimsize\vert\,#2}
		\newcommand{\cPr}[2]{\operatorname{Pr}\cPrParen{#1}{#2}}
		\newcommand{\E}[2]{\operatorname{E}_{#1}\left[#2\right]}
		\newcommand{\cE}[3]{\E{#1}{\left.#2\right|#3}}
		\newcommand{\Var}[2]{\operatorname{Var}_{#1}\left[#2\right]}
		\newcommand{\Cov}[2]{\operatorname{Cov}\left[#1,#2\right]}
		\newcommand{\CovMat}[1]{\operatorname{Cov}\left[#1\right]}

		%グラフ理論
		\newcommand{\neighborhood}{\mathcal{N}}

		%プログラミング
		\newcommand{\plpl}{\mathrel{++}}
		\newcommand{\pleq}{\mathrel{+}=}
		\newcommand{\asteq}{\mathrel{*}=}
