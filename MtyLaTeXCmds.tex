\usepackage{ifluatex}
\usepackage{listings, jvlisting}
\usepackage{mathtools}
\usepackage{xstring}

% Latex
    % The last definition of the command is applied across all pages.
    % see [Using later defined macro values](https://tex.stackexchange.com/questions/248076/using-later-defined-macro-values)
    \makeatletter
    % Reserve command name. The command must be pre-defined in advance of first use.
    % If you split the document into multiple files, you must put \rsvDef command in EACH of the file which refers the later-defined command.
    \newcommand*{\rsvDef}[1]{%
        \providecommand{#1}{%
            \begingroup%
                \def\tempString{The command `\string#1' is not fully defined in .aux file yet. Compile one more time.}%
                \textcolor{red}{\tempString}%
                \@latex@warning{\tempString}%
            \endgroup%
        }%
    }
    % Overriding command definition.
    \newcommand*{\laterDef}[2]{\protected@write\@auxout{}{\gdef\string#1{#2}}}
    \newcommand*{\laterEdef}[2]{%
        \begingroup%
            \edef\tempString{#2}%
            \protected@write\@auxout{}{\gdef\string#1{\tempString}}%
        \endgroup%
    }
    \makeatother

    % Obtain current head number.
    \newcommand*{\getTitleNum}{% "%" comments out the line-break
        \ifthenelse{\arabic{chapter}=0}{\thepart}{%
            \ifthenelse{\arabic{section}=0}{\thechapter}{%
                \ifthenelse{\arabic{subsection}=0}{\thesection}{%
                    \ifthenelse{\arabic{subsubsection}=0}{\thesubsection}{%
                        \ifthenelse{\arabic{subsubsubsection}=0}{\thesubsubsection}{\thesubsubsubsection}%
                    }%
                }%
            }%
        }%
    }

    % Escape underscore.
    % ---------- (deprecated at 2024-01-18) ----------
    % \makeatletter
    % \newcommand*{\@escapeUnderscore}[1]{#1\endgroup}
    % \newcommand*{\escapeUnderscore}{\begingroup\@makeother\_\@escapeUnderscore}
    % \makeatother
    % --------------------
    % ---------- (added at 2024-01-18) ----------
    \newcommand{\escapeUnderscore}[1]{\StrSubstitute{#1}{_}{\_\allowbreak}} % \allowbreak prevents overfull at table cell.
    % --------------------

    % new color for \textcolor command
    \definecolor{darkpastelgreen}{rgb}{0.01, 0.75, 0.24}

% 自然科学
    % 汎用
    \newcommand*{\ctext}[1]{\raise0.2ex\hbox{\textcircled{\scriptsize{#1}}}}

    % 数学
        % 汎用
        \DeclarePairedDelimiterX{\parens}[1]{\lparen}{\rparen}{#1}
        \DeclarePairedDelimiterX{\braces}[1]{\lbrace}{\rbrace}{#1}
        \DeclarePairedDelimiterX{\bracks}[1]{\lbrack}{\rbrack}{#1}
        \DeclarePairedDelimiterX{\verts}[1]{|}{|}{#1}
        \DeclarePairedDelimiterX{\Verts}[1]{\|}{\|}{#1}
        \newcommand*{\as}{{\quad\textrm{as}\quad}}
        \newcommand*{\st}{{\textrm{ s.t. }}}
        \DeclarePairedDelimiterX{\setComprehension}[2]{\lbrace}{\rbrace}{#1\,\delimsize\vert\,#2}
        \newcommand*{\naturalNumbers}{\mathbb{N}}
        \newcommand*{\integers}{\mathbb{Z}}
        \newcommand*{\rationalNumbers}{\mathbb{Q}}
        \newcommand*{\realNumbers}{\mathbb{R}}
        \newcommand*{\complexNumbers}{\mathbb{C}}
        \newcommand*{\field}{\mathbb{F}}
        \newcommand*{\func}[2]{{#1}\parens*{#2}}
        \newcommand*{\argmax}{\mathop{\textrm{arg~max}}}
        \newcommand*{\argmin}{\mathop{\textrm{arg~min}}}

        % 集合論
        \newcommand*{\range}[2]{\braces*{#1,\dotsc,#2}}
        \providecommand{\complement}{}\renewcommand{\complement}{\mathrm{c}}
        \newcommand*{\ind}[2]{\mathbbm{1}_{#1}\parens*{#2}}
        \newcommand*{\indII}[1]{\mathbbm{1}\braces*{#1}}

        % 数論
        \newcommand*{\abs}[1]{\verts*{#1}}
        \newcommand*{\combi}[2]{{_{#1}\mathrm{C}_{#2}}}
        \newcommand*{\perm}[2]{{_{#1}\mathrm{P}_{#2}}}
        \newcommand*{\GaloisField}[1]{\mathrm{GF}\parens*{#1}}

        % 解析学
        \newcommand*{\NapierE}{\mathrm{e}}
        \newcommand*{\sgn}[1]{\operatorname{sgn}\parens*{#1}}
        \newcommand*{\rect}[1]{\mathop{\mathrm{rect}}\parens*{#1}}
        \newcommand*{\cl}[1]{\operatorname{cl}#1}
        \newcommand*{\Img}[1]{\operatorname{Img}\parens*{#1}}
        \newcommand*{\dom}[1]{\operatorname{dom}\parens*{#1}}
        \newcommand*{\norm}[1]{\Verts*{#1}}
        \newcommand*{\floor}[1]{\left\lfloor#1\right\rfloor}
        \newcommand*{\ceil}[1]{\left\lceil#1\right\rceil}
        \newcommand*{\sinc}{\mathop{\mathrm{sinc}}}
        \newcommand*{\nsinc}{\mathop{\mathrm{nsinc}}}
        \newcommand*{\GammaFunc}[1]{\Gamma\parens*{#1}}
        \newcommand*{\erf}{\mathop{\mathrm{erf}}}
            % 逆三角関数
            \newcommand*{\asin}[1]{\operatorname{Sin}^{-1}{#1}}
            \newcommand*{\acos}[1]{\operatorname{Cos}^{-1}{#1}}
            \newcommand*{\atan}[1]{\operatorname{{Tan}^{-1}}{#1}}
            \newcommand*{\atanEx}[2]{\atan{\parens*{#1,#2}}}
            % 畳み込み
            \newcommand*{\cycConv}[2]{{#1}\underset{\text{cyc}}{*}{#2}}
            % 微分
            \newcommand*{\deriv}[3]{\frac{\operatorname{d}^{#3}#1}{\operatorname{d}{#2}^{#3}}}
            \newcommand*{\derivLong}[3]{\frac{\operatorname{d}^{#3}}{\operatorname{d}{#2}^{#3}}#1}
            \newcommand*{\partDeriv}[3]{\frac{\operatorname{\partial}^{#3}#1}{\operatorname{\partial}{#2}^{#3}}}
            \newcommand*{\partDerivLong}[3]{\frac{\operatorname{\partial}^{#3}}{\operatorname{\partial}{#2}^{#3}}#1}
            \newcommand*{\partDerivIIHetero}[3]{\frac{\operatorname{\partial}^2#1}{\partial#2\operatorname{\partial}#3}}
            \newcommand*{\partDerivIIHeteroLong}[3]{{\frac{\operatorname{\partial}^2}{\partial#2\operatorname{\partial}#3}#1}}
            % 積分
            \newcommand*{\PrincipalValue}{\mathrm{p.v.}}
            \newcommand*{\integrate}[5]{\int_{#1}^{#2}{#3}{\mathrm{d}^{#4}}#5}
            \newcommand*{\LebInteg}[4]{\int_{#1} {#2} {#3}\parens*{\mathrm{d}#4}}

        % 複素解析
        \newcommand*{\conj}[1]{\overline{#1}}
        \providecommand{\Re}{}\renewcommand{\Re}[1]{{\operatorname{Re}{\parens*{#1}}}}
        \providecommand{\Im}{}\renewcommand{\Im}[1]{{\operatorname{Im}{\parens*{#1}}}}
        \newcommand*{\Arg}[1]{\operatorname{Arg}{\parens*{#1}}}
        \newcommand*{\Log}[1]{\operatorname{Log}{#1}}
            % Laplace 変換
            \newcommand*{\LPLC}[1]{\operatorname{\mathcal{L}}\parens*{#1}}
            \newcommand*{\ILPLC}[1]{\operatorname{\mathcal{L}}^{-1}\parens*{#1}}
            % 離散 Fourier 変換
            \newcommand*{\DFT}[1]{\mathrm{DFT}\parens*{#1}}
            % Z 変換
            \newcommand*{\ZTrans}[1]{\operatorname{\mathcal{Z}}\parens*{#1}}
            \newcommand*{\IZTrans}[1]{\operatorname{\mathcal{Z}}^{-1}\parens*{#1}}

        % 線形代数
        \newcommand*{\bm}[1]{{\boldsymbol{#1}}}
        \newcommand*{\matEntry}[3]{#1\bracks*{#2}\bracks*{#3}}
        \newcommand*{\matPart}[5]{\matEntry{#1}{#2:#3}{#4:#5}}
        \newcommand*{\diag}[1]{\operatorname{diag}\parens*{#1}}
        \newcommand*{\transpose}[1]{{#1}^\top}
        \newcommand*{\HerConj}[1]{{#1}^*}
        \newcommand*{\tr}[1]{\operatorname{tr}{\parens*{#1}}}
        \newcommand*{\inprod}[2]{\left\langle#1,#2\right\rangle}
        \newcommand*{\HadamardProd}{\odot}
        \newcommand*{\HadamardDiv}{\oslash}
        \newcommand*{\Span}[1]{\operatorname{span}\bracks*{#1}}
        \newcommand*{\Ker}[1]{\operatorname{Ker}\parens*{#1}}
        \newcommand*{\rank}[1]{\operatorname{rank}\parens*{#1}}
            % ベクトル
                % 単位ベクトル
                \newcommand*{\vix}{\bm{i}_x}
                \newcommand*{\viy}{\bm{i}_y}
                \newcommand*{\viz}{\bm{i}_z}

        % 確率論
        \newcommand*{\PDF}[2]{\operatorname{PDF}\bracks*{#1,\;#2}}
        \newcommand*{\Ber}[1]{\operatorname{Ber}\parens*{#1}}
        \newcommand*{\Beta}[2]{\operatorname{Beta}\parens*{#1,#2}}
        \newcommand*{\ExpDist}[1]{\operatorname{ExpDist}\parens*{#1}}
        \newcommand*{\ErlangDist}[2]{\operatorname{ErlangDist}\parens*{#1,#2}}
        \newcommand*{\PoissonDist}[1]{\operatorname{PoissonDist}\parens*{#1}}
        \newcommand*{\GammaDist}[2]{\operatorname{Gamma}\parens*{#1,#2}}
        \newcommand*{\cind}[2]{\ind{#1\left| #2\right.}} % 条件付き指示関数
        \providecommand{\Pr}{}\renewcommand{\Pr}[1]{\operatorname{Pr}\parens*{#1}}
        \DeclarePairedDelimiterX{\cPrParens}[2]{(}{)}{#1\,\delimsize\vert\,#2}
        \newcommand*{\cPr}[2]{\operatorname{Pr}\cPrParens{#1}{#2}}
        \newcommand*{\E}[2]{\operatorname{E}_{#1}\bracks*{#2}}
        \newcommand*{\cE}[3]{\E{#1}{\left.#2\right|#3}}
        \newcommand*{\Var}[2]{\operatorname{Var}_{#1}\bracks*{#2}}
        \newcommand*{\Cov}[2]{\operatorname{Cov}\bracks*{#1,#2}}
        \newcommand*{\CovMat}[1]{\operatorname{Cov}\bracks*{#1}}

        % グラフ理論
        \newcommand*{\neighborhood}{\mathcal{N}}

    % プログラミング
    \newcommand*{\plpl}{\mathrel{++}}
    \newcommand*{\pleq}{\mathrel{+}=}
    \newcommand*{\asteq}{\mathrel{*}=}
    \ifluatex
        \newcommand*{\inlineCode}[1]{\lstinline[basicstyle=\ttfamily]|#1|}
    \else
        \newcommand*{\inlineCode}[1]{\lstinline[inputencoding=utf8, basicstyle=\ttfamily]|#1|}
    \fi
